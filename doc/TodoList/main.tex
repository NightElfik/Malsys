\documentclass[12pt, a4paper]{article}

\usepackage{comment}
\usepackage{fancyhdr}
\usepackage[czech]{babel}
\usepackage[T1]{fontenc}
\usepackage[utf8x]{inputenc}
\usepackage[pdftex]{graphicx}
\usepackage{hyperref}
\usepackage[all]{hypcap}
\usepackage{indentfirst}	% ident first paragraphs too
\usepackage{anysize}	% \marginsize command
\usepackage{subfig}
\usepackage{url}
\usepackage{color}


\pagestyle{fancyplain}
\marginsize{2cm}{2cm}{2cm}{2cm}

\title{Malsys -- TODO list}
\author{Marek Fišer}
\date{26. 6. 2011}


\begin{document}

\maketitle

Tento seznam představuje to, co vše by se dalo udělat. Není to specifikace.


%------------------------------------------------------------------------------
\section{Obecně}
\begin{itemize}
	\item lokalizace -- vše v AJ nebo uvažovat nad lokalizacemi
\end{itemize}


%------------------------------------------------------------------------------
\section{Knihovna pro generování L-systému}
DLL knihovna

\subsection{Parser vstupu}
\begin{itemize}
	\item návrh syntaxe zápisu L-sysému, důraz na čitelnost a snadnou rozšiřitelnost
		
		syntaxe by měla být nezávislá na netisknutelných znacích, a tím pádem i na formátování
	
	\item podpora UTF (pravděpodobně UTF-8)
	
	\item něco jako JavaDoc u předpisů pro jednoduchý komentář parametrů, autorů, zdrojů, atd.

	\item datové struktury pro reprezentaci \uv{syrových} (naparsovaných) dat, měly by umožňovat definovat L-systém přímo, bez nutnosti parsovat (měly by probíhat kontroly na korektnost hodnot)
	
	\item tokenizer + parser v F\# (FsLex a FsYacc), který bude parsovat přímo do datových struktur
	
	\item součástí parseru bude část, která bude parsovat aritmetické výrazy, u ní důraz na rozšiřitelnost (nové operátory, funkce)
	
	\item parser by měl co nejlépe a nejpřesněji ohlásit kde je syntaktická chyba, měl by být zotavitelný (nezastaví se na první chybě, ale chybnou oblast přeskočí a parsuje dál -- snazší ladění syntaktických chyb pro uživatele)
	
	\item reverzní parser (vypisovač) -- dostane struktury a vrátí text L-systému
	
		jeho součástí by mohl být i formátovač, který dostane validní kód a ten jednotně zformátuje
\end{itemize}

\subsection{Vyhodnocování L-systému}
\begin{itemize}
	\item návrh jednotlivých nezávislých komponent (a jejich rozhraní) pro generování L-systému -- přepisovací systém by měl být nezávislý na renderování, což umožní mít mnoho různých (různě specializovaných a optimalizovaných) komponent

		každá komponenta by měla mít na starosti jen jednu (jednoduchou) činnost a tu poskytovat přes rozhraní ostatním komponentám, aby se o ní nemusely starat komponenty, které se o ní starat nechtějí -- renderer chce kreslit čáry, ne počítat si dle symbolů kde je právě poloha \uv{želvy}

		na druhou stranu návrh musí podporovat určitou rozšiřitelnost, např. symboly můžou mít definovány instrukce, kterým bude rozumět jen uživatelem definovaná komponenta, ostatní komponenty mu k nim musí dát přístup

		návrh by měl podporovat také zařazení dalších komponent do celku (jakési pluginy), např. enviromentální modul
	\item implementace základních přepisovacích systémů s podporou podmíněných, kontextových a stochastických přepisovacích pravidel
	\item implementace základních rendererů pro render do bitmapy a do SVG
\end{itemize}


%------------------------------------------------------------------------------
\section{Desktopová GUI aplikace}
takový \uv{chytřejší} textový editor

\subsection{Rozhraní}
\begin{itemize}
	\item zvážit jakou technologii použít (pravděpodobně WinForms vs WPF)
	
	\item návrh GUI
	
	\item úzká spolupráce s parserem, syntax highlighting, přehledné zobrazování syntaktických chyb, formátování kódu
	
	\item důraz na pohodlnou práci (aby se v poznámkovém bloku nepsalo lépe), tzn. textový editor musí obsahovat všechny obvyklé funkce, zejména zpět/vpřed
	
	\item možnost práce jak přímo s textem tak pomocí klikacích výběrů -- to by mělo zejména sloužit pro přehled všech možností, které jdou v daném kontextu nastavit
	
	\item živé náhledy (pro pohodlné ladění L-systému) s nastavitelnými parametry
	
	\item FASS designer (mřížka)
\end{itemize}

\subsection{Renderování}
\begin{itemize}
	\item pomocí renderovacího serveru
	
	\item možnost ovládání fronty rendereovacího serveru pomocí GUI, zobrazování dostupných údajů o tom co se právě renderuje
	
	\item i po ukončení GUI editoru může pokračovat renderování L-systémů
\end{itemize}


%------------------------------------------------------------------------------
\section{Renderovací server}
konzolová aplikace

\begin{itemize}
	\item umožní zařazování úkolů k renderování do fronty s nastavitelnými parametry
	
	\item k samotnému renderování používá knihovnu
	
	\item ovládání pomocí konzole (textové příkazy)
	
	\item rozumná paralelizace vykonávání jednotlivých úkolů (počet paralelně běžících není větší než počet jader procesoru)
	
	\item prioritní fronta -- některé úkolů se díky prioritě spočítají dříve (to je užitečné např. když běží výpočet několika L-systémů, který trvá velice dlouho a uživatel chce vypočítat jeden malý a nechce čekat)
	
	\item možnost ovládat frontu (odebírat a řadit úkoly)
	
	\item různé politiky renderování, např. kolik prostředků si brát, zda renderovat jen když uživatel nepracuje (jako spořič)
	
	\item timeout pro úkol (jak dlouho se snažit počítat)
	
	\item zvážit zda neudělat komunikaci se serverem přes síť (TCP/IP), to by mohl být dobrý základ pro snadné oddělení výpočetně náročné části od ovládání

také by se pak dalo uvažovat o rozkládání zátěže mezi více serverů, které mezi sebou mohou komunikovat a předávat si úkoly
\end{itemize}


%------------------------------------------------------------------------------
\section{Web}
jednoduché webové rozhraní pro možnost generování L-systémů z webu

\begin{itemize}
	\item rozhodnout technologii (pravděpodobně PHP vs ASP.NET)
	
	\item jednoduché rozhraní pro render L-systému
	
	\item mnoho ukázek + dokumentace jak používat (ne nutně vše-pokrývající), důraz na jednoduchost, srozumitelnost
	
	\item malé L-systémy se zobrazí přímo (timeout pár sekund)
	
	\item možnost zaslání výsledku na mail -- jediná možnost jak vygenerovat větší L-systémy
	
	\item příjímání úkolů přes mail
	
	\item úzká spolupráce s render serverem
	
	\item historie renderovaných předpisů
\end{itemize}


%------------------------------------------------------------------------------
\section{Ostatní}
\begin{itemize}
	\item enviromentální modul, jeho základ (datové struktury) vypracuji jako zápočťák na předmět Aplikovaná Geometrie
\end{itemize}





\end{document}

























