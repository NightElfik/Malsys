
\section{Web user interface}
\label{sec:design-web}

The user interface is a very important part of the whole project.
Two basic forms of the user interface were considered: desktop application and web site.
The web was chosen because for the following reasons.

\begin{description*}
	\item[Accessibility]
		The Web is accessible on a wide range of operating systems where desktop applications cannot be ported easily.
		Besides the usual desktop systems it is possible to browse it on mobile devices such as smart phones or tablets.		
	\item[No installation]
		The end-user does not have ti install anything: the application does not depend on the user's OS. \nomenclature{OS}{operating system}
		The solution is not easy to setup because it has many dependencies to third-party libraries.
		The Web application is installed by experienced an administrator and everything is then set up properly.
	\item[Community]
		Users can share and discuss their work at the same place where they create it.
		This helps to create a community which is important to all projects.
	\item[Up to date]
		The web user interface is always up to date.
		All updates are instantly applied and available to all users.
		Errors can be logged and the administrator can fix them as soon as possible.		
\end{description*}

\begin{wrapfigure}{r}{0.4\textwidth}
	\vspace{-20pt}
	\includegraphics[width=\linewidth]{Sunflower}
	\caption{Logo of the web}
	\label{fig:logo}
	\vspace{-20pt}
\end{wrapfigure}

The web user interface also serves as a comprehensive example of the \lsystem processing library and its use and capabilities.
The Sunflower model in \autoref{fig:logo} was produced by the web site and because of its shape, which fits into a rectangle and its good recognizability even as a $32 \times 32$ pixels image, it was chosen as the logo of the web page.

The web page has four main parts.
The first three parts, namely the \emph{\lsystem processor}, \emph{Gallery of \lsystems} and \emph{Help} are accessible to everyone.
The fourth part is the \emph{Administration} and it is accessible only to administrators.


\subsection{\lsystem processor}

The main functionality of the web is the processing of a user's input (source code) and showing the results.
For this purpose there is a web page with a big text area where the source code can be written.
There are three possibilities how to submit the source code.

The first is processing the source code and showing all the results (or list of errors).
If there are too many outputs they are packed into one ZIP file.
All the results can be downloaded.

The second possibility is to just compile the source code and see the compiled source code (no results are shown).
This is intended for the debugging of errors in the input.

The last possibility, which is only available for registered users, is to save the source code.
To be able to save the source code successfully it must be without compilation errors.
For each saved source code a unique identifier is generated and with it is possible to access the saved input (by a permanent link).
Saved inputs can be published in gallery.


\subsection{Gallery of \lsystems}

The gallery will serve as a showcase of the capabilities of \lsystems for new users as well as learning material.
All entries in the gallery will have their source code included and anybody can try to process and customize it.
Registered users can rate other gallery entries.

Every registered user may contribute to the gallery with their own creation.
To enter an \lsystem into the gallery a user has to save and publish the source code via the \lsystem processor.
It is possible to alter the thumbnail \lsystem over the original \lsystem.
This allows the simplification of images into thumbnails and show complex models in the detail view.

Tags can be assigned to each \lsystem in the gallery.
A tag is a short keyword, term or abbreviation which helps to describe the \lsystem and allows it to be found again.
A list of all tags can be listed and a list of \lsystems filtered by a specific tag can be shown.
A tag can contain a short description of its meaning.
The description can be edited only by a special user group.

\lsystems can also be filtered by user name.


\subsection{Help}
\nomenclature{FAQ}{frequently asked questions}

An important part of the web is the help section.
Help contains a few basic topics and FAQs (frequently asked questions) for new users.
Then there is list of predefined components, process configurations, constants, functions and operators.
The last part of the help is a detailed syntax reference.


\subsection{Administration}

The administration section of the web is accessible to a restricted group of users.
There is more than one administrators group, every one with different privileges.

The main administrators group is able to manage roles for all users, manage user groups (roles) and explore error logs.

The next group is able to explore all processed inputs on the site, see all saved inputs and export the input database to a text file.

The last group can see a list of the submitted feedbacks and if a new feedback is submitted all users from this group will receive it via e-mail.


\subsection{Database}

The database will serve for saving all necessary data.
Figures \ref{fig:dbSchema1} and \ref{fig:dbSchema2} shows the database scheme (\emph{PK} after name means primary key and \emph{FK} foreign key).


\tikzstyle{db} = [draw, fill=blue!12, rectangle split, rectangle split parts=2]
\begin{figure}[p]
	\centering
	\begin{tikzpicture}[auto,>=latex,shorten >=2pt]
		\node (user) [db, text width=6cm] {\textbf{User} \nodepart{second}
			UserId [Int32] (PK)\\
			Name [String] \\
			NameLowercase [String] \\
			PasswordHash [Binary] \\
			PasswordSalt [Binary] \\
			Email [String] \\
			RegistrationDate [DateTime] \\
			LastLoginDate [DateTime] \\
			LastActivityDate [DateTime] \\
			LastPwdChangeDate [DateTime]
		};
			
		\node (feedback) [db, below of=user, node distance=7cm, text width=6cm] {\textbf{Feedback} \nodepart{second}
			FeedbackId [Int32] (PK) \\
			UserId [String] (FK) \\
			Subject [String] \\
			SubmitDate [DateTime] \\
			Email [String] \\
			Message [String] \\
			IsNew [Bool]
		};
			
		\node (role) [db, above of=user, node distance=6cm, text width=6cm] {\textbf{Role} \nodepart{second}
			RoleId [Int32] (PK) \\
			Name [String] (FK) \\
			NameLowercase [String]
		};		
		
		\node (x) [coord, above right of=user, node distance=8cm] {};
		
		\node (vote) [db, above of=x, node distance=4cm, text width=6cm] {\textbf{Saved input vote} \nodepart{second}
			SavedInputId [Int32] (PK) \\
			UserId [Int32] (PK) \\
			Rating [Int32]
		};
		
		
		\draw [->] (feedback) -- (user)  node[pos=0.2]{*}  node[pos=0.55]{UserId}  node[pos=0.85]{0..1};
		\draw [shorten >=0] (role) -- (user) node[pos=0.2]{*}  node[pos=0.8]{*};
		\draw [shorten >=0] (vote) -- (x) node[pos=0.2]{*};
		\draw [->] (x) -- (user) node[pos=0.5]{UserId}  node[pos=0.8]{1};
		
		\draw [->] (x.east) -- +(3cm,0)  node[below,pos=0.5]{SavedInputId}  node[pos=0.8]{1}   node [xshift=1.5cm] {Saved inputs};
		\draw [<-,shorten >=0,shorten <=2pt] (user.east) ++(0,1cm) -- +(4cm,0) node[pos=0.2]{1}  node[below,pos=0.5]{CreationUserId}  node[pos=0.8]{*}   node [xshift=1.5cm] {Saved inputs};
		\draw [<-,shorten >=0,shorten <=2pt] (user.east) ++(0,-2cm) -- +(4cm,0) node[pos=0.2]{0..1}  node[below,pos=0.5]{UserId}  node[pos=0.8]{0..1}   node [xshift=1.5cm] {Input process};
		
	\end{tikzpicture}
	\caption{First half of the database scheme of the web}
	\label{fig:dbSchema1}
\end{figure}

\begin{figure}[p]
	\centering
	\begin{tikzpicture}[auto,>=latex,shorten >=2pt]
		\node (input) [db, text width=6.5cm] {\textbf{Saved inputs} \nodepart{second}
			SavedInputId [Int32] (PK)\\
			UrlId [String] \\
			ParentInputProcessId [Int32] (FK) \\
			CreationUserId [Int32] (FK) \\
			CreationDate [DateTime] \\
			EditDate [DateTime] \\
			IsPublished [Bool] \\
			IsDeleted [Bool] \\
			PublishName [String] \\
			Views [Int32] \\
			SourceSize [Int32] \\
			OutputSize [Int64] \\
			Duration [Int64] \\
			MimeType [String] \\
			SourceCode [String] \\
			ThumbnailSourceExtension [String] \\
			Description [String] \\
			OutputMetadata [Binary] \\
			OutputThnMetadata [Binary] \\
			RatingSum [Int32] \\
			RatingCount [Int32]
		};
			
		\node (tag) [db, left of=input, node distance=7.5cm, text width=4.5cm] {\textbf{Tag} \nodepart{second}
			TagId [Int32] (PK) \\
			Name [String] \\
			NameLowercase [String] \\
			Description [String]
		};
			
		\node (proc) [db, below of=input, node distance=10cm, text width=6.3cm] {\textbf{Input process} \nodepart{second}
			InputProcessId [Int32] (PK) \\
			ParentInputProcessId [Int32] (FK) \\
			ChainLength [Int32] \\
			CanonicInputId [Int32] (FK) \\
			UserId [Int32] (FK) \\
			ProcessDate [DateTime] \\
			Duration [Int64]
		};
		
		\node (output) [db, below of=proc, node distance=6cm, text width=6cm] {\textbf{Process output} \nodepart{second}
			ProcessOutputId [Int32] (PK) \\
			InputProcessId [Int32] (FK) \\
			FileName [String] \\
			CreationDate [DateTime] \\
			LastOpenDate [DateTime] \\
			Metadata [Binary]
		};
		
		\node (canonic) [db, xshift=-1cm,  below left of=proc, node distance=8.5cm, text width=5.3cm] {\textbf{Canonic input} \nodepart{second}
			CanonicInputId [Int32] (PK) \\
			Hash [Int32] \\
			SourceCode [String] \\
			SourceSize [DateTime] \\
			OutputSize [DateTime]
		};
		
		
		\draw [shorten >=0] (input) -- (tag)  node[pos=0.2]{*}  node[pos=0.8]{*};
		\draw [->] (input) -- (proc)  node[pos=0.2]{*}  node[pos=0.5]{ParentInputProcessId}  node[pos=0.8]{0..1};
		\draw [->] (output) -- (proc)  node[pos=0.2,right]{*}  node[pos=0.5,right]{InputProcessId}   node[pos=0.8,right]{0..1};
		\draw [->] (proc) -- (canonic)  node[pos=0.08]{*}  node[pos=0.35]{CanonicInputId}   node[pos=0.65]{1};
		\draw [->] (proc) edge [in=180,out=190,loop] node[pos=0.15]{0..1}  node[pos=0.5]{ParentInputProcessId}   node[pos=0.9]{*} ();
		
		\draw [<-,shorten >=0,shorten <=2pt] (input.west) ++(0,4cm) -- +(-4cm,0) node[above,pos=0.2]{1}  node[below,pos=0.5]{SavedInputId}  node[above,pos=0.8]{*}   node [xshift=-2cm] {Saved input vote};
		\draw [->] (input.west) ++(0,-4cm) -- +(-4cm,0) node[above,pos=0.2]{1}  node[below,pos=0.5]{CreationUserId}  node[above,pos=0.8]{*}   node [xshift=-1cm] {User};
		\draw [->] (proc.west) ++(0,1cm) -- +(-4cm,0) node[above,pos=0.2]{0..1}  node[below,pos=0.5]{UserId}  node[above,pos=0.8]{0..1}   node [xshift=-1cm] {User};
	\end{tikzpicture}
	\caption{Second half of the database scheme of the web}
	\label{fig:dbSchema2}
\end{figure}


In the left part of the scheme shown in \autoref{fig:dbSchema1} are the tables \emph{User} and \emph{Roles} with the relation \emph{n} to \emph{n} (any user can be in any number of roles).
Both tables table contains column called \emph{NameLowercase} for canonical representation of the user names for easier searching.
The \emph{Feedback} table for saving posted feedbacks have a foreign key to \emph{Users} (if a registered user submits a feedback).

The right part of the scheme shown in \autoref{fig:dbSchema2} is more complicated.
Every processed input is saved to the \emph{Input process} table.
To optimize the size of the database the source code is not saved for every input process but it is canonicalized and saved to the \emph{Canonic input} table.
The hash is counted for every saved canonical input to speed up lookup for identical inputs.
This system ensures that two identical source codes will not be saved in the database.
One might thing that the probability of processing two identical source codes is very low but it is not true.
The most users trying to process the \lsystems from  the gallery and do minor changes to them like changing the number of iterations.

The results of processing (like images) are not saved directly into the database because may be relatively large.
The result are saved to the hard disk to the \emph{working directory} (which can be configured).
Each produced file is saved into the \emph{Process output} table to ensure effective cleaning of the physical files in the working directory.

\emph{LastOpenDate} entry (in the \emph{Process output} table) is updated every time the user views processed file.
If the number of stored files exceeds the maximum (which can be configured) the files with the longest time before last opening are deleted.
This mechanism allows to keep alive "old but viewed" files with no need for saving them permanently (for example for sharing with a friend).

Moreover, the new files are saved with the \emph{LastOpenDate} lowered by one minute over the \emph{CreationDate}.
This will cause that deletion of non-viewed files is likely than the viewed ones.
It can protect wiping all files by some bot that will process many inputs but do not open them.

Lets go back to the saving of all processed inputs to the \emph{Input process} table.
The creation of an \lsystem is iterative process.
At the beginning is simple \lsystem which is gradually improved by the user.
To keep track over this iterative process the "parent" input processes is saved for each processed input (if any exists).

The processes forms chains.
The longer the chain of processes is, the better \lsystem can be expected.
The length of the chain can be counted by searching the database an resolving the \emph{ParentInputProcessId} column.
However this process can take very long time because the \emph{Input process} table will likely have many rows.
To be possible to easily find the longest chain the chain length is counted for each row (column \emph{ChainLength}).

If new input is about to save to the \emph{Input process} table and it do not have the parent (for example the first process after opening the page), the \emph{Canonic input} table is searched for corresponding input.
If the canonical input it is found, the oldest\footnote{More input process entries can share one canonical input entry.} corresponding input process is selected as the parent.


\subsubsection{Saved inputs}

Registered users can save their inputs.
Inputs are saved to the \emph{Saved inputs} table which also serves as table for the gallery.
For every saved input is generated unique ID stored in the \emph{UrlId} column.
The ID is used in the permanent link which allows permanent access to all saved inputs.

The saved inputs can be edited by the owner but, more importantly, they can be published to the gallery.
The inputs in the gallery can be rated.
Ratings are stored in the \emph{Saved input vote} table.
The primary key to this table is a pair of the \emph{SavedInputId} and \emph{UserId} allowing each user to vote for every input just once (of course the the vote can be changed later).

The published entries in the \emph{Saved inputs} table are sorted by average rating taking into account total number of the votes (the more votes, the better).
To speed up the sorting and eliminate joining with the \emph{Saved input vote} table, the sum and the count of votes is stored directly in the \emph{Saved inputs} table.

The source code of saved inputs is saved as is (without any canonicalization) to preserve the comments and formatting.
The last output of processed \lsystem is saved as the result (image or thumbnail) which allows to generate the thumbnails effectively by adding the thumbnail source extension (the \emph{ThumbnailSourceExtension} column) to the end of the actual source code (the \emph{SourceCode} column).
In the thumbnail source extension can be used the process statement to generate thumbnail easily.






































