
\section{Web user interface}

The user interface is very important part of whole project.
To basic forms of user interface was considered, desktop application and web site.
The web was chosen because of following reasons.

\begin{description*}
	\item[Accessibility]
		Web is accessible on wide range of operating systems where desktop application can not be ported easily.
		Besides usual desktop systems it is possible to browse it on mobile devices like smart phones or tablets.		
	\item[No installation]
		End-user do not install anything, the application does not depend on user's OS.
		The solution is not easy to setup because it has many dependencies to third-party libraries.
		Web application is installed by experienced administrator and everything is set up properly.
	\item[Community]
		Users can share and discuss their work on the same place where they create it.
		This helps to create community which is important to all projects.
	\item[Up to date]
		Web user interface is always up to date.
		All updates are instantly applied for all users.
		Errors can be logged and administrator can fix them as soon as possible.		
\end{description*}

\begin{wrapfigure}{r}{0.4\textwidth}
	\vspace{-20pt}
	\includegraphics[width=\linewidth]{Sunflower}
	\caption{Logo of the web}
	\label{fig:logo}
	\vspace{-20pt}
\end{wrapfigure}

Web user interface also serve as comprehensive example of \lsystem processing library and its usage and capabilities.
Sunflower model in \autoref{fig:logo} was produced by the web site and because of its shape which fits in rectangle and good recognizability even as $32 \times 32$ pixels image it was chosen as the logo of the web page.

Web page have four main parts. First three parts, namely \emph{\lsystem processor}, \emph{Gallery of \lsystems} and \emph{Help} are accessible to anyone.
Fourth part is the \emph{Administration} and it is accessible only to administrators.

\subsection{\lsystem processor}

Main functionality of the web is processing of user's input (source code) and showing results.
For this purpose there is web page with big text area where the source code can be written.
There is three possibilities how to submit the source code.

First is processing source code and showing all results (or list of errors).
If there are too many outputs they are packed to one ZIP file.
All results can be downloaded.

Second possibility is to just compile source code and see compiled source code (no results are showed).
This is intended for debugging of errors in the input.

Last possibility which is available only for registered users is to save the source code.
To be able to save the source code successfully it must be without compilation errors.
For each saved source code unique is generated and it can be accessed by permanent link.
Saved inputs can be published in gallery.


\subsection{Gallery of \lsystems}

The gallery will serve as showcase of capabilities of \lsystems for new users as well as learning material.
All entries in gallery will have their source code included and anybody can try to process and customize it.
Registered users can rate other gallery entries.

Every registered user may contribute to gallery with their own creation.
To \lsystem into gallery user have to save and publish source code via \lsystem processor.
It is possible to alter thumbnail \lsystem over original \lsystem.
This allows to simplify images in thumbnail and show complex model in detail.

Tags can be assigned to each \lsystem in gallery.
Tag is short keyword, term or abbreviation which helps to describe \lsystem and allows it to be found again.
List of all tags can be listed and list of \lsystems filtered by specific tag can be shown.
Tag can contain short description of its meaning.
The description can be edited only by special user group.

\lsystems can be filtered also by user name.


\subsection{Help}

Important part of the web is the help.
Help contains few basic topics and FAQ (frequently asked questions) for new users.
Then there is list of predefined components, process configurations, constants, functions and operators.
Last part of the help is detailed syntax reference.


\subsection{Administration}

Administration section of the web is accessible to restricted group of users.
There is more administrators group every with different privileges.

The main administrators group is able to manage roles for all users, manage user groups (roles) and explore error logs.

Next group is able to explore all processed inputs on site, see all saved inputs and export input database to text file.

Last group can see list of submitted feedbacks and if the new feedback is submitted all users from this group will receive it via e-mail.




















