
\section{Web user interface}

[TODO]
... example of usage of designed library.

\subsection{Advantages and disadvantages}

Web user interface have many advantages over desktop application.
The biggest advantage is no need of installation with is connected with no requirements of user operation system and its version, installed runtime environment and libraries.
User do not have to install and configure anything and start using it immediately.
Web can be displayed even on mobile devices like smart phones or tablets.

Next big advantage is concentration of community.
Users can share and discuss their work on the same place where they create it.

Web user interface is always up-to date.
All updates are instantly applied for all users.
All errors can be logged and administrator can fix them as soon as possible.
Configuration of whole system is done and tested by administrators with excludes compatibility errors of library.

However web user interface have also disadvantages.
The first and the most obvious disadvantage is need of connection to internet.
Also connection speed can play significant role in quality of work.
Large images or models can be loaded slowly.
Web have to be adapted to the most of web browser to display content right.


\subsection{Processing of input}

Main functionality of the web will be processing user's input source code and showing results.
For this purpose there will be web page with big text area where input source code can be written.
After submitting the source code all results will be listed if processing was successful or list of errors if not.
If there will be too many outputs they will be packed to one file.
All results will be available for download and some of them which can be display will be displayed.


\subsection{Gallery}

Important part of the web will be gallery of results.
The gallery will serve as showcase of capabilities of \lsystems for new users as well as learning material.
All entries in gallery will have their source code included and anybody can try to process it and alter it.

Every registered user may contribute to gallery with their own creation.
Registered users can also rate other gallery entries.


\subsection{Help}

Last major part of the web will be help.
Help will contain basic topics and FAQ (frequently asked questions) for new users and reference of all content.
In help will be many examples and hyperlinks to related topics.


\subsection{Administration}

Administration section of the web will be accessible to restricted group of users.




















