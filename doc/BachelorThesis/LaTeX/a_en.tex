\documentclass[12pt,a4paper]{report}
\setlength\textwidth{145mm}
\setlength\textheight{247mm}
\setlength\oddsidemargin{10mm}
\setlength\evensidemargin{0mm}
\setlength\topmargin{0mm}
\setlength\headsep{0mm}
\setlength\headheight{0mm}

\usepackage[czech,english]{babel}
\usepackage{fontspec}

\usepackage{xspace}

\newcommand{\lsystem}{\mbox{L-system}\xspace}
\newcommand{\lsystems}{\mbox{L-systems}\xspace}


\begin{document}
\pagestyle{empty}

An \lsystem in its simplest form is a variant of a context-free grammar.
Originally, \lsystems were developed and are still mainly used for modeling plant growth, though with \lsystems it is possible to create general fractals, models of towns or even music.
However, anyone interested in \lsystems and wanting to experiment with them may have difficulty finding an appropriate application.
The goal of this work was to create an online system, suitable for a wide range of users, for working and experimenting with \lsystems.
The resulting solution consists of two parts.

The first part is a universal, easily-expandable library for processing \mbox{L-sys}\-tems.
Expandability is achieved thanks to its modularity.
All input is processed through interconnected components that are specialized in particular activities.
The specialization of the components also contributes to the clarity and reliability of the whole processing system.
The library is independent and multiplatform and can thus be readily used in other applications.

The second part consists of a modern web interface designed to be understandable for beginners and yet also capable of providing advanced features for more advanced users.
Part of the site is a gallery of \lsystems to which each user can contribute and which thus helps to create a user-community.
The web interface takes full advantage of the library and thus serves as an example of its use.


\end{document}




































