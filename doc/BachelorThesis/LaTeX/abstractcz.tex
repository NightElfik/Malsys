
%L-systémy, které byly vynalezeny již v roce 1968 Aristidem Lindenmayerem za účelem modelování rostlin se ujaly a používají se dodnes.
%Mohli jste se s nimi setkat například ve slavném filmu Avatar, ve kterém mnoho stromů a rostlin bylo vytvořeno právě za pomocí L-systémů.
%Jejich využití se však neomezuje jen na rostliny, ale používají se například pro skládání hudby nebo pro modelování ulic ve virtuálních městech.

%Pokud někoho L-systémy osloví a chce s nimi experimentovat má několik možností.
%Jedna možnost je vyzkoušet nějaký web nebo webový applet.
%Ty jsou většinou velice jednoduché a nenabízí příliš možností.
%Další možností je stáhnout si nějaký program, který nabízí více možností.
%%K němu je ale třeba nastudovat dokumentaci, pochopit ovládání a syntaxi vstupu.
%Samostatný program zase ztrácí výhodu webové stránky, která je dostupná odkudkoliv a nabízí místo pro komunitu.
%%Problém samostatných programů je v tom, že je těžké najít jiné uživatele používající stejný program se kterými bsyte mohli sdílet výsledky své práce nebo diskutovat o společných problémech.

%Tato práce se snaží zkombinovat to nejlepší z výše zmíněných možností.
%Cílem této práce je vytvořit webovou stránku, která nabídne uživatelům snadno přístupné rozhraní pro experimentování s L-systémy.
%Rozhraní by mělo být srozumitelné pro nováčky a zároveň by mělo nabízet pokročilé funkce pro náročnější uživatele.
%Tento cíl se podařilo splnit.
%Výsledný webový generátor podporuje celou řadu grafických výstupů jako 2D a 3D obrázky nebo prostý text.
%Součástí webu je i galerie L-systémů do které může každý uživatel přispívat a tvořit tak komunitu.

%Protože webový generátor podporuje širokou škálu výstupů, bylo třeba vytvořit univerzální systém pro generování L-systémů.
%Díky tomu vznikla modulární knihovna pro zpracování L-systémů, která se dá použít i samostatně bez webového rozhraní.
%Její síla spočívá v rozložení procesu zpracování do tzv. komponent, každá komponenta má na starosti malou část zpracování.
%Celý systém vznikne propojením komponent.
%Díky systému komponent je velice jednoduché knihovnu rozšiřovat, protože stačí vytvořit jen rozšiřující komponentu a zapojit ji do systému.

\section*{\lsystems online}
Marek Fišer

\section*{Omluva}
Chtěl bych se Vám omluvit za to, že úvod hovoří o něčem jiném než absrtakt.
Na textu pracuji od pondělka ale teprve v neděli jsem zjistil, co je vlastně mým cílem a čím jsem přispěl (problém implementačních prací).
Abstrakt jsem upravit stihl ale úvod už ne.
\\
\\
Mohl bych Vás poprosit, abyste si tuto práci nechala na opravování mezi posledními?
Pokusil bych se nedostatky odstranit a poslat lepší verzi v týdnu. Děkuji.
\\
\\
Abstrakt jsem napsal v čj a úvod v aj, ale to doufám není takový prohřešek.


\section*{Abstrakt}


Cílem této práce bylo vytvořit online systém pro práci s L-systémy a jejich vizualizaci.
Řešení se skládá ze dvou částí.

První část je univerzální knihovna pro zpracování L-systémů.%, která je navíc také multiplatformní.
Tato knihovna zpracovává vstup prostřednictvým úzce specializovaných komponent orientovaných na svou konkrétní činnost, což přispívá k přehlednosti a spolehlivosti celku.
Díky tomu je knihovna snadno rozšiřitelná o další funkcionalitu.
%Tato knihovna zpracovává vstup pomocí systému komponent.
%Komponenty jsou úzce specializované na jednu konkrétní činnost což přispívá k přehlednosti a spolehlivosti celku.
%Díky systému komponent je knihovna snadno rozšiřitelná o další funkcionalitu.
Navíc je knihovna zcela nezávislá a použitelná i v jiných aplikacích.

Druhá část je webové rozhraní, používající moderní technologie jako HTML5 WebGL pro vizualizaci L-systémů.
Rozhraní bylo navrženo tak, aby bylo srozumitelné pro nováčky a zároveň aby nabízelo pokročilé funkce pro náročnější uživatele.
Součástí webu je i galerie L-systémů do které může každý uživatel přispívat a tvořit tak komunitu.
Webové rozhraní plně využívá schopnosti navržené knihovny a slouží tak i jako ukázka jejího použití.





























