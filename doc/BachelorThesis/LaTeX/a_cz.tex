\documentclass[12pt,a4paper]{report}
\setlength\textwidth{145mm}
\setlength\textheight{247mm}
\setlength\oddsidemargin{10mm}
\setlength\evensidemargin{0mm}
\setlength\topmargin{0mm}
\setlength\headsep{0mm}
\setlength\headheight{0mm}

\usepackage[czech,english]{babel}
\usepackage{fontspec}


\begin{document}
\pagestyle{empty}

\mbox{L-systém} je v nejjednodušší podobě varianta bezkontextové gramatiky.
Byl vyvi\-nut a používá se hlavně pro modelování růstu rostlin, ale s jeho pomocí se také dají vytvářet obecné fraktály, modely měst nebo dokonce hudba.
Pokud někoho \mbox{L-systémy} zaujmou a chce s nimi experimentovat, je těžké najít aplikaci, která by mu to umožňovala.
Cílem této práce bylo vytvořit online systém pro práci a experimentování s L-systémy pro široké spektrum uživatelů.
Výsledné řešení se skládá ze dvou částí.

První část je univerzální, snadno rozšiřitelná knihovna pro zpracování \mbox{L-sys}\-témů.
Svou rozšiřitelnost dosahuje modularitou, vstup zpracovává pros\-třednic\-tvím systému propojených komponent, které jsou specializované na kon\-krét\-ní činnost.
To také přispívá k přehlednosti a spolehlivosti celku.
Navíc je knihovna zcela nezávislá a multiplatformní, lze ji tedy použít i v jiných aplikacích.

Druhá část je moderní webové rozhraní, které bylo navrženo tak, aby bylo srozumitelné pro nováčky a zároveň aby nabízelo pokročilé funkce pro nároč\-nější uživatele.
Součástí webu je i galerie L-systémů, do které může každý uživatel přispívat a tvořit tak komunitu.
Webové rozhraní plně využívá schopnosti navr\-žené knihovny a slouží tak i jako ukázka jejího použití.


\end{document}




































