
\chapter{Input syntax reference}
\label{chap:syntax}

This appendix contains formal definition of the designed syntax.
The syntax is described with the regular expressions which are explained in the first section.
The syntax is described from the most general parts to more concrete parts.

\section{Regular expressions}
\label{sec:regexps}

Table \ref{tbl:regexpExplanation} explains the syntax of the regular expressions used for description of the input syntax.

\begin{table}[ht]
	\centering
	\begin{tabular}{c p{160pt} p{170pt}}
   		\toprule
   		Regexp & Definition & Example \\
   		\midrule
		\texttt{' '} & matches the text between quotes (and nothing else) & \texttt{'let'} matches only string \texttt{let} \\ \hline
		\texttt{[ ]} & matches one of any characters enclosed in brackets & \texttt{[ab]} matches only \texttt{a} or \texttt{b} \\ \hline
		\texttt{[ - ]} & matches single character between two specified characters (inclusive) & \texttt{[0-9]} matches any digit from \texttt{0} to \texttt{9} \\ \hline
		\texttt{|} & matches regexp on the right \emph{OR} on the left of pipe & \texttt{'gray'|'grey'} matches only \texttt{gray} or \texttt{grey} \\ \hline
		\texttt{?} & preceding regexp must match zero or one times & \texttt{'colo' 'u'? 'r'} matches only \texttt{color} or \texttt{colour} \\ \hline
		\texttt{+} & preceding regexp must match one or more times & \texttt{[0-9]+} matches any non-negative integer like \texttt{5} or \texttt{42} \\ \hline
		\texttt{*} & preceding regexp must match zero or more times & \texttt{'b' 'e'*} matches \texttt{b}, \texttt{be} or \texttt{beee} \\
		\bottomrule
	\end{tabular}
	\caption{Meaning of syntax of regular expressions}
	\label{tbl:regexpExplanation}
\end{table}


\section{Tokens}

A token is atomic element of grammar.
In the token can not be any white-space character however the white-space characters are often used to separate individual tokens.
Names of the tokens will be upper-case to distinguish them from the grammar rules.


\subsection{Identifier}
\begin{Grammar}
ID = (ALPHA_CHAR | '_') (ALPHA_CHAR | DIGIT | '_')* '\''*
\end{Grammar}

The \texttt{ID} token represents identifier which starts with alphabetic character (letter) or underscore and may also contain digits.
At the end can be apostrophes to allow identifiers such as \texttt{a'} or \texttt{a''}.

Note that regular expression is simplified by the \emph{ALPHA\_CHAR} and \emph{DIGIT} to avoid using characters groups in unicode.
The ALPHA\_CHAR matches any letter and the DIGIT matches any digit character.


\subsection{Number}
\begin{Grammar}
NUMBER = [0-9]+ ('.' [0-9]+)? ([eE] ('+'|'-')? [0-9]+)?
	| '0'[bB] [01]+
	| '0'[oO] [0-7]+
	| '0'[xX] ([0-9] | [a-f] | [A-F])+
	| '#' ([0-9] | [a-f] | [A-F])+
\end{Grammar}

The \texttt{NUMBER} token represents a number literal.
Numbers can be specified in five formats: floating-point, binary, octal and hexadecimal prefixed \texttt{0x} or \texttt{\#}.


\subsection{Operator}
\begin{Grammar}
OPERATOR = (firstOpChar opChar*) | '==' | '/'

firstOpChar = [!$%&\<>@^|~:-] | '?' | '+' | '*'
opChar = firstOpChar | '=' | '/'
\end{Grammar}%$

The \texttt{OPERATOR} token represents an operator in mathematical expression.



\section{Input syntax}

The syntax is described with regular expressions explained in \autoref{sec:regexps}.
The regular expressions can contain literals, tokens or other regular expressions.
The input grammar is white-space independent, between any two regular expression members can be any number of white-space characters.

In each subsection, the first line of formal specification is the regular expression for described statement.
Next lines are describing regular expressions used in the main definition.


\subsection{Input}
\begin{Grammar}
input = inputStatement*

inputStatement = emptyStatement
	| constantDef
	| functionDef
	| lsystemDef
	| processStatement
	| processConfigDef
\end{Grammar}

The \emph{Input} rule is the start rule of the input syntax.
The Input can contain constant, function and \lsystem definitions, process statements and process configuration definitions.
Empty statement allows redundant semicolons between statements.


\subsection{Empty statement}

\begin{Grammar}
emptyStatement = ';'
\end{Grammar}

Empty statement allows redundant semicolons in syntax.


\subsection{Constant definition}
\begin{Grammar}
constantDef = 'let' ID '=' expression ';'
\end{Grammar}

Defines a constant with name represented by \texttt{ID} with value represented by \texttt{expression}.


\subsection{Function definition}
\begin{Grammar}
functionDef = 'fun' ID paramsDefValListParens
	'{' constantDef* 'return' expression ';' '}'
\end{Grammar}

Defines a function with name represented by \texttt{ID}, parameters (with optional default values) \texttt{paramsDefValListParens},
	local constants \texttt{constantDef} and return value \texttt{expression}.


\subsection{\lsystem definition}
\begin{Grammar}
lsystemDef = 'abstract'? 'lsystem' ID paramsDefValListParens?
	baseLsystems? '{' lsystemStatement* '}'

baseLsystems = 'extends' baseLsystemsList
baseLsystemsList = ID exprListParens? (',' baseLsystemsList)?
lsystemStatement = emptyStatement
	| constantDef
	| functionDef
	| componentPropertyAssign
	| symbolsInterpretationDef
	| rewriteRule
\end{Grammar}

Defines an \lsystem with name represented by \texttt{ID}, optional parameters (with optional default values) \texttt{paramsDefValListParens},
	base \lsystems \texttt{baseLsystems} and \lsystem statements \texttt{lsystemStatement}.
Arguments can be supplied to each base \lsystem.
The \lsystem statement can be constant, function and symbols interpretation definition, component property assign and rewrite rule.


\subsubsection{Component property assign}
\begin{Grammar}
componentPropertyAssign = 'set' ID '=' expression ';'
	| 'set' 'symbols' ID '=' symbolExprArgs* ';'
\end{Grammar}

Defines a component property assign of property with name represented by \texttt{ID} to value represented by \texttt{expression} (value properties)
	or to list of symbols \texttt{symbolExprArgs} (symbol properties). 


\subsubsection{Symbols interpretation definition}
\begin{Grammar}
symbolsInterpretationDef = 'interpret'
	symbol+ paramsDefValListParens? 'as' ID exprListParens? ';'
\end{Grammar}

Defines an interpretation method with name represented by \texttt{ID} to symbols \texttt{symbol}.
If parameters \texttt{paramsDefValListParens} are specified values of arguments of interpreted symbols are matched to parameters and they can be used as variables in the interpretation method arguments \texttt{exprListParens}.


\subsubsection{Rewrite rule definition}
\begin{Grammar}
rewriteRule = 'rewrite' rrPattern rrConsts? rrCondition?
	'to' rrReplacement ';'
\end{Grammar}

Defines a rewrite rule for symbol (and its context) specified in \texttt{rrPattern} to symbols \texttt{rrReplacement}.
Optionally there can be specified local constants \texttt{rrConsts} and condition \texttt{rrCondition}.


\subsubsection{Rewrite rule pattern}
\begin{Grammar}
rrPattern = rrContext? symbolOptParams rrContext?

symbolOptParams = SYMBOL symbol_params?
symbol_params = '(' ( ID (',' ID)* )? ')'
rrContext = '{' symbolPptParams* '}'
\end{Grammar}

Defines a pattern of the rewrite rule which defines rewriting of a symbol represented by \texttt{symbolOptParams}.
The first \texttt{rrContext} represents left context of main symbol \texttt{symbolOptParams} and the second \texttt{rrContext} represents right context.
Main symbol and every symbol in context can have specified parameters names.
Actual arguments of matched symbols will be set to specified parameters.


\subsubsection{Rewrite rule constants definition}
\begin{Grammar}
rrConsts = 'with' rrCostDefsList

rrCostDefsList = ID '=' expression  (',' rrCostDefsList)?
\end{Grammar}

Defines local variables in the rewrite rule separated by comma.
Syntax is similar to the constant definition but there is no \emph{let} keyword at the beginning and no semicolon at the end.


\subsubsection{Rewrite rule condition}
\begin{Grammar}
rrCondition = 'where' expression
\end{Grammar}

Defines a rewrite rule condition.


\subsubsection{Rewrite rule replacement}
\begin{Grammar}
rrReplacements = 'nothing'
	| symbolExprArgs* rrWeight? ('or'? 'to' rrReplacements)?

rrWeight = 'weight' expression
\end{Grammar}

Defines one or more replacements for the rewrite rule.
Each replacement can have probability weight.



\subsubsection{\lsystem symbol}
\begin{Grammar}
symbol = ID | OPERATOR | '[' | ']' | '.'
symbolExprArgs = symbol exprListParens?
\end{Grammar}


\subsection{Process configuration definition}
\begin{Grammar}
processConfigDef = 'configuration' ID '{' processConfigStatement* '}'

processConfigStatement = emptyStatement
	| procConfComponentDef
	| procConfContainerDef
	| procConfConnectionDef
\end{Grammar}

Defines a process configuration with name represented by \texttt{ID} with statements \texttt{processConfigStatement}.
Statements can be component, container or connection definition.


\subsubsection{Process configuration component definition}
\begin{Grammar}
procConfComponentDef = 'component' ID 'typeof' typeId ';'
\end{Grammar}

Defines a component with name represented by \texttt{ID} with type \texttt{typeId}.


\subsubsection{Process configuration container definition}
\begin{Grammar}
procConfContainerDef =
	'container' ID 'typeof' typeId 'default' typeId ';'
\end{Grammar}

Defines a container with name represented by \texttt{ID} with type \texttt{typeId} (first) with default component with type \texttt{typeId} (second).


\subsubsection{Process configuration connection definition}
\begin{Grammar}
procConfConnectionDef = 'virtual'? 'connect' ID 'to' ID '.' ID ';'
\end{Grammar}

Defines a connection of component with name represented by \texttt{ID} (first) to property with name \texttt{ID} (third) of component with name \texttt{ID} (second).


\subsection{Process statement}
\begin{Grammar}
processStatement = 'process' name exprListParens?
	'with' ID useComponents* ';'

name = 'all' | ID
useComponents = 'use' ID 'as' ID
\end{Grammar}

Defines processing of one or \emph{all} \lsystems represented by \texttt{name} with parameters \texttt{exprListParens} with process configuration \texttt{ID}.
At the end or process statement can be specified usage of extra components in containers in configuration.



\subsection{Mathematical expression}

\begin{Grammar}
expression = exprMember+

exprMember = NUMBER
	| ID
	| OPERATOR
	| exprIndexer
	| exprArray
	| exprFunction
	| '(' expression ')'
exprIndexer = '[' expression ']'
exprArray = '{' exprList? '}'
exprFunction = ID exprListParens
\end{Grammar}

The expression consists of list of members.
The \texttt{ID} token represents a variable.
Meaning of the rest of members are obvious from their names.
The grammar of expression is not strict, correctness of the expression will ensure the compiler.
The parser can not parse expression as tree because the operators are not defined while parsing.



\subsection{Common rules}

\subsubsection{Mathematical expression list}
\begin{Grammar}
exprList = expression (',' expression)*
exprListParens = '(' exprList? ')'
\end{Grammar}


\subsubsection{List of parameters with default values}
\begin{Grammar}
paramsDefValList = ID ('=' expression)?  (',' paramsDefValList)?
paramsDefValListParens = '(' paramsDefValList? ')'
\end{Grammar}


\subsubsection{Type identifier}
\begin{Grammar}
typeId = ID ('.' ID)*
\end{Grammar}

Represents a fully qualified type identifier.









