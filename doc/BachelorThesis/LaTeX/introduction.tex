
\chapwithtoc{Introduction}

\lsystem (also called Lindenmayer system) is mathematical formalism developed for plant growth modeling by Aristid Lindenmayer in 1968~\cite{Lin68}.
In the simplest form an \lsystem is variant of regular or \mbox{context-free} grammar.
By rewriting (deriving) initial string of symbols (also called axiom) with rewrite rules from grammar \lsystem produces string of symbols which can be interpreted in many different ways.
In first \lsystems by Lindenmayer were symbols interpreted as cells of algae.
A different approach was taken by Przemyslaw Prusinkiewicz who interpreted \lsystem symbols with \mbox{Logo-like} turtle\footnote{
	Logo is computer programming language developed for use in education of programming for children.
	Logo controls cybernetic turtle which is drawing on 2D canvas.}~\cite{Pru85}.
With this method he obtained more plant-like structures and fractals.~\cite{CD93}

Over the time \lsystems were used in many areas.
\lsystems were used to generate rivers in fractal mountains~\cite{PH93}, streets in virtual of cities~\cite{PM01} and to describe subdivision curves~\cite{PSSK03}.
\lsystems can be used in other fields than computer graphics for example in music generation~\cite{HCJ99, Man06}.
They are still used in plant modeling.
Plant models generated with \lsystems are used in modern video games or films, for example they were used to generate many plants and trees for famous film Avatar~\cite{Wor08, Dun10}.
%\citeauthor{SBM10} presented automatic generation of \lsystems from 2D model~\cite{SBM10}.

\lsystems has wide variety of interesting applications but it is hard to find some place to experiment with them.
Web-based \lsystem generators are often too primitive and do not offer nothing more than generation of simple fractals.
Most of desktop applications are specialized for some concrete \lsystem application and not easy to control.
Also they may not be compatible with operating system you are using and some of them are not free.
Goal of this work is to fill this gap and create online development environment for anybody who wants to experiment with \lsystems.
Development environment will be divided into two parts, web user interface and \lsystem processing library.

User interface will be web site which offers great accessibility, anybody around the world can use it from any computer.
Interface should be friendly to new users and also offer advanced features for experienced users.
Primary output format of web based \lsystem processor will be 2D images (vector and raster) but it will also be possible to create and display 3D outputs using modern HTML5 WebGL\footnote{
	WebGL (Web-based Graphics Library) is a cross-platform, royalty-free web standard for a low-level 3D graphics API based on OpenGL ES 2.0, exposed through the HTML5 Canvas element as Document Object Model interfaces.
	WebGL code executes on a computer display card's GPU (Graphics Processing Unit).} technology directly in browser.
Part of the web site will be gallery of \lsystems.
Any registered user can add his creation to gallery along with some description and others can rate it.
This will help to create community of active users and it can also serve as learning tool for new users.

Second part of work will be \lsystem processing library.
Although it will be designed to support demands of web interface, it will be independent and it should be possible to use it in other application.
During design of library great emphasis will be placed on easy extensibility to make the it as universal as possible.
It should be possible to extend library with user written plugins.

New syntax for input will be designed to improve user experience especially for new users.
It will be clean, easy to learn and remember.
The syntax will cover all needs of library from defining \lsystems to configuring whole processing system.
This will also ensure that whole input can be written in one file which will simplify source code sharing and saving.
Parser generator will be used for creating robust and extensible parser.


\section*{Structure}

In the first chapter is be formally defined \lsystems and described their extending features.
Second chapter is dedicated to the design of whole solution, subsections are describing individual parts.
Third chapter is about implementation of designed system and it has similar structure as second chapter.


All examples of \lsystems in this work will be in designed syntax and they can be tried on web.
See syntax reference for details [??].































