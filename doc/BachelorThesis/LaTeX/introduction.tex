
\chapwithtoc{Introduction}

\lsystem (also called Lindenmayer system) is mathematical formalism developed for plant growth modeling by Aristid Lindenmayer in 1968~\cite{Lin68}.
In the simplest form an \lsystem is variant of regular grammar.
Using rewriting rules from grammar \lsystem produces string of symbols which can be interpreted in many different ways.
In first \lsystems by Lindenmayer were symbols interpreted as cells of algae.
A different approach was taken by Przemyslaw Prusinkiewicz who interpreted \lsystem symbols with Logo-like turtle\footnote{
	Logo is computer programming language developed for use in education of programming for children.
	Logo controls cybernetic turtle which is drawing on 2D canvas.}~\cite{Pru85}.
With this method he obtained more plant-like structures and fractals.~\cite{CD93}

Over the time \lsystems were used in other areas.
\lsystems were used to generate rivers in fractal mountains~\cite{PH93}, street generating in procedural modeling of cities~\cite{PM01} and to describe subdivision curves \cite{PSSK03}.
\lsystems can be used in other fields than computer graphics for example in music generation~\cite{HCJ99, Man06}.
They are still used in plant modeling, for example they were used to generate many plants and trees to famous film Avatar~\cite{Wor08, Dun10}.
\citeauthor{SBM10} presented automatic generation of \lsystems from 2D model~\cite{SBM10}.

\lsystems has wide variety of interesting applications but it is hard to find some place to experiment with them.
Web-based \lsystem generators are often too primitive and most of desktop applications are specialized for some concrete \lsystem application.
Goal of this work is to create online development environment for anybody who wants to experiment with \lsystems.
Design of development environment will be divided into two parts, user interface and \lsystem processing library.

User interface will be web site which offers great accessibility, anybody can visit it from any computer around the world.
Interface should be friendly to new users and also offer advanced features for experienced users.
Primary output format of web based \lsystem processor will be 2D images (vector and raster) but it will be capable to create and display 3D outputs with modern HTML5 WebGL\footnote{
	WebGL (Web-based Graphics Library) is a cross-platform, royalty-free web standard for a low-level 3D graphics API based on OpenGL ES 2.0, exposed through the HTML5 Canvas element as Document Object Model interfaces.
	WebGL code executes on a computer display card's GPU (Graphics Processing Unit).} directly in browser.
Part of the web site will be gallery of \lsystems.
Any registered user can add his creation to gallery along with some description.
It can also serve as learning tool for new users.

New syntax for \lsystem input will be created to improve user experience especially for new users.
The syntax should be clean, easy to learn and remember.

\lsystem processing library will be created to support demands of web interface.
Emphasis will be placed on extensibility.
...

All examples of \lsystems in this work will be in designed syntax and can be processed by library or tried on web.
See syntax reference for details [??].


%Each application uses \lsystem to generate string of symbols and than symbols are processed to get desired output.
%But each application needs little different \lsystem features.
%It is common practice that each application will also implement \lsystem processing part at its own even if.
%An idea of universal system for \lsystem generation comes to mind.
%The system which will offer to use \lsystem common features to specific application.
%The universal system should allow user to focus on implementing theirs specific application using \lsystem features from universal system.

%This work starts with these ideas in mind.
%The goal is to implement universal and extensible system for working with \lsystems.
%The system should be capable to cover as wide variety of applications as possible.
%It should be possible to easily use common features of \lsystems as well as to redefine them.

%An example of usage and capabilities of the universal system is as important part as design and implementation itself.
%As an example will be created web page.
%It will serve as detailed demonstration and also it will allow users around the world to try and play with \lsystems.

%Emphasis will be given on easy extensibility

%It is very hard and time-consuming for human to create realistic model of tree or even whole forest.
%It is easier to write \lsystem rules specific to desired species of tree and than generate whole forest of trees in which no two trees are identical.






































