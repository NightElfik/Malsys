
\chapter{Third-party libraries and services}
\label{chap:thirdParty}

\section{F\# PowerPack}
\label{sec:FSharpPowerPack}
\srcurl{http://fsharppowerpack.codeplex.com/}

\noindent
The F\# PowerPack is a collection of libraries and tools for use with the F\# programming language provided by the F\# team at Microsoft.
The PowerPack includes F\# versions of lexer and parser generation tools (FsLex and FsYacc), along with the MSBuild tasks to incorporate them in the build process.

The FsLex and FsYacc are used for parsing of input (see \autoref{sec:parsingImplementaion}).


\section{HTML5 boilerplate}
\label{sec:HTML5boilerplate}
\srcurl{http://html5boilerplate.com/}

\noindent
HTML5 Boilerplate is the professional frontend developers's base HTML/CSS/JS template for a fast, robust and future-safe site.

It is used as the base of the HTML and CSS in the web user interface.


\section{Three.js}
\label{sec:ThreeJs}
\srcurl{http://github.com/mrdoob/three.js/}

\noindent
Three.js is lightweight JavaScript 3D library (3D engine) for rendering 3D scenes directly in web browser.

It is used to render 3D models produced by the \hyperref[Malsys.Processing.Components.Renderers.ThreeJsSceneRenderer3D]{\emph{ThreeJsSceneRenderer3D}} component.


\section{jQuery}
\label{sec:jQuery}
\srcurl{http://jquery.com/}

\noindent
jQuery is a fast and concise JavaScript Library that simplifies HTML document traversing, event handling, animating, and Ajax interactions for rapid web development.

All custom JavaScript win web user interface was written with the help of jQuery.


\section{Modernizr}
\label{sec:Modernizr}
\srcurl{http://modernizr.com/}

\noindent
Modernizr is an open-source JavaScript library that helps to build the next generation of HTML5 and CSS3-powered websites.

Cool HTML5 and CSS3 features can be used in web user interface thanks to Modernizr.


\section{Code Contracts}
\label{sec:CodeContracts}
\srcurl{http://research.microsoft.com/en-us/projects/contracts/}

\noindent
Code Contracts provide a language-agnostic way to express coding assumptions in .NET programs.
The contracts take the form of preconditions, postconditions, and object invariants.
Contracts act as checked documentation of your external and internal APIs.
The contracts are used to improve testing via runtime checking, enable static contract verification, and documentation generation.

The Code Contracts are used in many methods across all projects.
Contract checking is turned on while the solution is built on the debug settings but because of performance reasons they are turned off in the release.

In \autoref{cs:contracts} are defined four preconditions checking validity of arguments and two postconditions ensuring output format.
From the contracts is clear that the method must be supplied with at least 3 points in list and the number of returned indices will be divisible by 3 (without reminder).
It makes sense because every 3 indices specifies a triangle.


\begin{Csharp}[label=cs:contracts,caption={Example of code contracts in the \emph{Triangularize} method of the \emph{Polygon3DTrianguler} class.}]
public List<int> Triangularize(IList<Point3D> points,
		Polygon3DTriangulerParameters prms) {
		
	Contract.Requires<ArgumentNullException>(points != null);
	Contract.Requires<ArgumentNullException>(prms != null);
	Contract.Requires<ArgumentException>(points.Count >= 3);
	Contract.Requires<ArgumentException>(
		prms.TriangleEvalDelegate != null);
		
	Contract.Ensures(Contract.Result<List<int>>() != null);
	Contract.Ensures(Contract.Result<List<int>>().Count % 3 == 0);
	...	
}
\end{Csharp}



\section{Autofac IoC container}
\label{sec:autofac}
\srcurl{http://code.google.com/p/autofac/}

\noindent
Autofac is an IoC container for Microsoft .NET.
It manages the dependencies between classes so that applications stay easy to change as they grow in size and complexity.
This is achieved by treating regular .NET classes as components.

The Autofac have the ASP.NET MVC 3 integration which extends the Autofac by helper methods for simple registration of controllers.


\section{MvcContrib}
\label{sec:mvcContrib}
\srcurl{http://mvccontrib.codeplex.com/}

\noindent
MvcContrib project adds functionality and ease-of-use to Microsoft's ASP.NET MVC Framework.

The part of the MvcContrib is \emph{T4MVC} which is a T4 template for ASP.NET MVC apps that creates strongly typed helpers that eliminate the use of literal strings when referring the controllers, actions and views (see \autoref{sec:implT4mvc}).

The MvcContrib contains many useful UI components like the \emph{Grid} which is used for displaying data as table (\autoref{code:adminGrid}).

\begin{Razor}[label=code:adminGrid,caption={Usage of the \emph{Grid} component to show list of user roles}]
@Html.Grid(Model).Columns(col => {
		col.For(x => x.RoleId)
			.Named("Id")
			.HeaderAttributes(style => "width: 40px;")
			.Attributes(@class => "center");
		col.For(x => x.Name);
		col.For(x => Html.ActionLink("Edit", MVC.Administration.Roles.Edit(x.RoleId)))
			.Attributes(@class => "center")
			.HeaderAttributes(style => "width: 50px;")
			.Encode(false);
	}).Attributes(@class => "w100")
\end{Razor} 


\section{Elmah}
\label{sec:elmah}
\srcurl{http://code.google.com/p/elmah/}

\noindent
ELMAH (Error Logging Modules and Handlers) is an application-wide error logging facility that is completely pluggable.
It can be dynamically added to a running ASP.NET web application, or even all ASP.NET web applications on a machine, without any need for re-compilation or re-deployment.
Once ELMAH has been dropped into a running web application and configured appropriately it is possible to log nearly all unhandled exceptions.

For usage of ELMAH in the web project see \autoref{sec:implErrorLog}

 
\section{LESS css}
\label{sec:lesscss}
\srcurl{http://lesscss.org/}

\noindent
LESS is dynamic stylesheet language.
It extends CSS with dynamic behavior such as variables, mixins, operations and functions.
LESS can run on both the client-side and server-side.

LESS was used for definition of stylesheets in the web (see \autoref{sec:implLess}).

\subsection{.LESS}
\srcurl{http://www.dotlesscss.org/}

\noindent
.LESS (pronounced dot-less) is a .NET port of the LESS JavaScript libary.
It allows implicit compilation of LESS files to CSS.


\section{Data Annotations Extensions}
\label{sec:dataAnnoExt}
\srcurl{http://dataannotationsextensions.org/}

\noindent
Data Annotations Extensions provides common validation attributes which extend the built-in ASP.NET DataAnnotations. 
The core library provides server-side validation attributes that can be used in any .NET 4.0 project.

For the the usage of the annotations see \autoref{sec:implDataAnn}.



\section{Yahoo! UI Library}
\label{sec:yuiLib}
\srcurl{http://yuicompressor.codeplex.com/}

\noindent
Yahoo! UI Library is a .NET port of the Yahoo! UI Library's YUI Compressor Java project.
The objective of this project is to compress any Javascript and Cascading Style Sheets to an efficient level that works exactly as the original source, before it was minified.

The Yahoo! UI Library is used for minimalization of JavaScript files to improve site performace \autoref{sec:implJs}. 


\section{ReCaptcha}
\label{sec:reCaptcha}
\srcurl{http://recaptcha.net/}

\noindent
reCAPTCHA is a free CAPTCHA\footnote{A CAPTCHA is a program that can tell whether its user is a human or a computer.} service that helps to digitize books, newspapers and old time radio shows.

reCAPTCHA is used at user registration and feedback pages to prevent spamming by bots.



\section{Google Analytics}
\label{sec:ga}
\srcurl{http://www.google.com/analytics/}

\noindent
Google Analytics (GA) is a free service offered by Google that collects visitors data and generates detailed statistics about them.
The GA can track visitors from all referrers, including search engines, social networks, etc.

The GA is used for tracking the activity of users in the web user interface.





















