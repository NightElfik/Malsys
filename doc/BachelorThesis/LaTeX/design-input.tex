
\section{Input design}

Input is very important part of application.
\lsystems have no standardized input for example like programming languages.
Every implementation of \lsystem generator uses its own syntax.
Main goals for input design are simplicity and extensibility.
With simple input there is lower barrier for new user to start using the application.
Complicated input might discourage many potential users.

There are two basic types of input, graphic interface or source code.
Each of them have its advantages and disadvantages.

Graphic interface is better for new users.
Values are written in text boxes and chosen from select boxes so it easier to create valid input.
Also parsing of input is easier because values are separated by user.
On the other hand implementation of well-arranged graphic interface for complex system can be very hard and time-consuming.

Source code is better for saving, sharing and versioning.
Statements can be easily commented thus ideas behind code can be saved with it.
Parts of code can be copy-pasted and the syntax can be easily extended.
Parsing of source code is significantly harder but input interface can be just one text area.

Source code was chosen as input because of its advantages over graphic interface.
To achieve simplicity of input source code the syntax will be rich on keywords.
This should ensure that even new users will understand the statements.

For robust and extensible parsing of input will be used third-party parser generator.



\section{Source code syntax}

Source code will allow to define \lsystems and also configure whole processing system.

The syntax will be white-space independent which means that white space characters can be used for formatting.


\subsubsection{Constant definition}


















