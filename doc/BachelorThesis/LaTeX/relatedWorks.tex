
\section{Related \lsystem generators}

In this section will be listed other computer programs or web pages that allows to process \lsystems and eventually interpret them in most cases as an image.

\subsection{Web based}
\label{sec:WebBasedGenerators}

\subsubsection{\lsystem generator by Michael Norris}
{ \vspace{-10pt} \footnotesize \url{http://www.michaelnorris.info/software/l-system-generator.html} }

Simple script which offers to set basic properties of L-system namely number of iterations, axiom and up to 15 rewrite rules.
Result is list of strings of symbols from all iterations (it does not interpret symbols).


\subsubsection{Lindenmayer power by MadFlame Software}
{ \vspace{-10pt} \footnotesize \url{http://madflame991.blogspot.com/p/lindenmayer-power.html} }

L-system generator which offers to set basic properties of L-system and interpretation for each symbol.
Symbols can be interpreted as turtle graphics or they can define or modify value of variable.
All iterations are listed as text and drawn on screen as well.


\subsubsection{\lsystem generator by Nolan Carroll}	
{ \vspace{-10pt} \footnotesize \url{http://nolandc.com/sandbox/fractals/} }

L-system generator with quite nice interface where is possible to set basic properties of L-system and interpretation.
Interpretation of symbols is unchangeable.
Last iteration of L-system is drawn on screen.


\subsubsection{VRML \lsystem generator by Patrick Murris}
{ \vspace{-10pt} \footnotesize \url{http://www.alpix.com/vrml/lsys.htm} }
  
L-system generator which can generate 3D VRML model.
Basic properties of L-system and interpretation can be set.
Only problem is that VRML plugin is needed for displaying 3D models.


\subsubsection{\lsystem generator by John Snyders}
{ \vspace{-10pt} \footnotesize \url{http://hardlikesoftware.com/projects/lsystem/lsystem.html} }
  
At first sight quite sophisticated L-system generator which can rewrite symbols with parameters and do context-sensitive rewriting.
Result is drawn on page as animation of development.
Biggest drawback is that L-systems are hard-coded in JavaScript and it is only possible to browse them in different iterations.



\subsection{Desktop applications}
\label{sec:DesktopGenerators}

\subsubsection{L-studio by Przemysław Prusinkiewicz et. al}
{ \vspace{-10pt} \footnotesize \url{http://algorithmicbotany.org/lstudio/} }

\begin{wrapfigure}{r}{0.5\textwidth}
	\vspace{-20pt}
	\begin{center}
	\includegraphics[width=0.48\textwidth]{Lily}
	\end{center}
	\caption{Model of Lily produced by L-studio}
\end{wrapfigure}

L-studio is one of the best applications for modeling plants with \lsystems (it is also possible to create general fractals with it).
L-studio is not single program but it is complex solution that consists of many tools.
With L-studio it is possible to model 3D models of plants with regards to environment like wind, gravity, space the around plant, sun light, etc.
Output can be saved in many formats like Wavefront OBJ, Postscript, BMP or render plant with built in ray-tracer to produce photo-realistic images.

Even there is many examples of plant models and extensive help it is not easy to start using it.
The syntax is very compact and quite unclear.

Application is not free-ware but demo version can be downloaded.
After evaluation period it is still possible to use it but it is not possible to export images and previews have watermark. 


\subsubsection{\lsystems explorer by James Matthews}
{ \vspace{-10pt} \footnotesize \url{http://www.generation5.org/content/2002/lse.asp} }
\label{sec:LsystemExplorer}

Simple desktop application which renders L-systems in window.
Basic properties of \lsystem and its interpretation can be edited in dialog window but interpretation for individual symbols can not be changed.
\lsystems can be saved or loaded into text file but drawn image can not be saved.


\subsubsection{\lsystem Vector Generator by Dmitry Malutin}
{ \vspace{-10pt} \footnotesize \url{http://xaraxtv.at.tut.by/lsvg.htm} }

Similar application to \nameref{sec:LsystemExplorer} but it is also possible to randomize line lengths or turn angles.
Nice feature is \emph{angle wizard} which displays grid \lsystems each with different setting of turn angle.
It is possible to save image as AI or WMF which are not most common formats.


\subsubsection{L-System 4 by Timothy Perz}
{ \vspace{-10pt} \footnotesize \url{http://www.oocities.org/tperz/L4About.htm} }

L-System 4 is quite advanced tool for generating models with \lsystems.
Besides all basic functionality it is possible to create 3D models with textures.
Models can be saved to raster images as BMP or JPEG or they can be exported to AutoCAD DXF format.
Interpreting capabilities are quite good but \lsystem rewriting can do only deterministic rewriting with limited usage of parameters.























