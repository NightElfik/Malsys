%
%%% Required information page of thesis
{%
\setlength\parindent{0mm}%
\setlength\parskip{5mm}%
Název práce: L-systémy online

Autor: Marek Fišer

E-maiolvá adresa autora: malsys@marekfiser.cz

Katedra: Kabinet software a výuky informatiky

Vedoucí bakalářské práce: RNDr. Josef Pelikán

E-maiolvá adresa vedoucího: pepca@cgg.mff.cuni.cz

Klíčová slova: Lindenmayerovy systémy, L-systémy, modelování, rostlin, systém komponent, SVG, WebGL

\section*{Abstrakt}
}
\mbox{L-systém} je v nejjednodušší podobě varianta bezkontextové gramatiky.
Byl vyvi\-nut a používá se hlavně pro modelování růstu rostlin, ale s jeho pomocí se také dají vytvářet obecné fraktály, modely měst nebo dokonce hudba.
Pokud někoho \mbox{L-systémy} zaujmou a chce s nimi experimentovat, je těžké najít aplikaci, která by mu to umožňovala.
Cílem této práce bylo vytvořit online systém pro práci a experimentování s L-systémy pro široké spektrum uživatelů.
Výsledné řešení se skládá ze dvou částí.

První část je univerzální, snadno rozšiřitelná knihovna pro zpracování \mbox{L-sys}\-témů.
Svou rozšiřitelnost dosahuje modularitou, vstup zpracovává pros\-třednic\-tvím systému propojených komponent, které jsou specializované na kon\-krét\-ní činnost.
To také přispívá k přehlednosti a spolehlivosti celku.
Navíc je knihovna zcela nezávislá a multiplatformní, lze ji tedy použít i v jiných aplikacích.

Druhá část je moderní webové rozhraní, které bylo navrženo tak, aby bylo srozumitelné pro nováčky a zároveň aby nabízelo pokročilé funkce pro nároč\-nější uživatele.
Součástí webu je i galerie L-systémů, do které může každý uživatel přispívat a tvořit tak komunitu.
Webové rozhraní plně využívá schopnosti navr\-žené knihovny a slouží tak i jako ukázka jejího použití.

\clearpage
{%
\setlength\parindent{0mm}%
\setlength\parskip{5mm}%
Title: \lsystems online

Author: Marek Fišer

Author's e-mail address: malsys@marekfiser.cz

Department: Department of Software and Computer Science Education

Supervisor: RNDr. Josef Pelikán

Supervisor's e-mail address: pepca@cgg.mff.cuni.cz

Keywords: Lindenmayer systems, \lsystems, plant, modelling, component system, SVG, WebGL

\section*{Abstract}
}
\lsystem is in the simplest form variant of context-free grammar.
\lsystems were developed and are used mainly for modeling of plant growth.
However with \lsystems it is possible to create general fractals, models of towns or even music.
If someone is interested in \lsystems and wants to experiment with them, it is difficult to find an application that would allow it.
The goal of this work was to create an online system for working and experimenting with \lsystems for a wide range of users.
The resulting solution consists of two parts.

The first part is a universal, easily expandable library for processing of \mbox{L-sys}\-tems.
The expandability is achieved with modularity.
The input is processed with interconnected components that are specialized in specific activities.
Specialization of components also contributes to the clarity and reliability of the whole processing system.
The library is independent and multiplatform therefore it can be used in other applications.

The second part is a modern web interface that was designed to be understandable for beginners and also to provide advanced features for more advanced users.
Part of the site is a gallery of \lsystems in which each user contribute which helps to create the community.
The web interface takes full advantage of the library and thus serve as an example of its use.




































