
%%% Required information page of thesis
{
\setlength\parindent{0mm}
\setlength\parskip{5mm}

Název práce: L-systémy online

Autor: Marek Fišer

E-maiolvá adresa autora: malsys@marekfiser.cz

Katedra: Kabinet software a výuky informatiky

Vedoucí bakalářské práce: RNDr. Josef Pelikán

E-maiolvá adresa vedoucího: pepca@cgg.mff.cuni.cz

Klíčová slova: Lindenmayerovy, systémy, modelování, rostlin

Abstrakt:
}
\mbox{L-systém} je v nejjednodušší podobě varianta bezkontextové gramatiky.
Byl vyvi\-nut a používá se hlavně pro modelování růstu rostlin, ale s jeho pomocí se také dají vytvářet obecné fraktály, modely měst nebo dokonce hudba.
Pokud někoho \mbox{L-systémy} zaujmou a chce s nimi experimentovat, je těžké najít aplikaci, která by mu to umožňovala.
Cílem této práce bylo vytvořit online systém pro práci a experimentování s L-systémy pro široké spektrum uživatelů.
Výsledné řešení se skládá ze dvou částí.

První část je univerzální, snadno rozšiřitelná knihovna pro zpracování \mbox{L-sys}\-témů.
Svou rozšiřitelnost dosahuje vysokou modularitou, vstup zpracovává pros\-třednic\-tvím systému propojených komponent, které jsou specializované na kon\-krét\-ní činnost.
To také přispívá k přehlednosti a spolehlivosti celku.
Navíc je knihovna zcela nezávislá a multiplatformní, lze ji tedy použít i v jiných aplikacích.

Druhá část je moderní webové rozhraní, které bylo navrženo tak, aby bylo srozumitelné pro nováčky a zároveň aby nabízelo pokročilé funkce pro nároč\-nější uživatele.
%Pro vizualizaci 3D \mbox{L-systémů} využívá moderní technologie jako HTML5 WebGL.
Součástí webu je i galerie L-systémů, do které může každý uživatel přispívat a tvořit tak komunitu.
Webové rozhraní plně využívá schopnosti navr\-žené knihovny a slouží tak i jako ukázka jejího použití.

\newpage

{
\setlength\parindent{0mm}
\setlength\parskip{5mm}

Title: L-systems online

Author: Marek Fišer

Author's e-mail address: malsys@marekfiser.cz

Department: Department of Software and Computer Science Education

Supervisor: RNDr. Josef Pelikán

Supervisor's e-mail address: pepca@cgg.mff.cuni.cz

Keywords: Lindenmayer, systems, plant, modelling

Abstract:
}



\begin{comment}
\vbox to 0.5\vsize{
\setlength\parindent{0mm}
\setlength\parskip{5mm}


Title:
L-systems online

Author:
Marek Fišer

Department:
Department of Software and Computer Science Education
% dle Organizační struktury MFF UK v angličtině

Supervisor:
RNDr. Josef Pelikán, pracoviště
% dle Organizační struktury MFF UK, případně plný název pracoviště
% mimo MFF UK v angličtině


Abstract:  % 80 - 200 words

Keywords:  % 3 to 5 keywords
L-system, library, web

\vss}\nobreak\vbox to 0.49\vsize{
\setlength\parindent{0mm}
\setlength\parskip{5mm}


Název práce:
L-systémy online

Autor:
Marek Fišer

Katedra:
Kabinet software a výuky informatiky

Vedoucí bakalářské práce:
RNDr. Josef Pelikán, pracoviště
% dle Organizační struktury MFF UK, případně plný název pracoviště mimo MFF UK


Abstrakt:  % 80 - 200 words


Klíčová slova:  % 3 to 5 keywords


\vss}
\end{comment}





































